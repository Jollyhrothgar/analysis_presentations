\section{From Last Time}
\label{Section1_FromLastTime}

\begin{frame}
\frametitle{From Last Time}
\begin{itemize}
	\item Explored various parameteterizations of the beam z-profile
	\item Fits show same results as data driven method, but they are wrong
	\item Simple gaussian model produces different results when used in Amaresh's
		framework vs my framework
	\item Machinery in place for rootfinding, minimization of differences between
		simulation and data.
\end{itemize}
\end{frame}
\begin{frame}
\frametitle{From Last Time}

\textbf{Homework:} \\
I was tasked to figure out why the simple gaussian model looks wrong - there
should be a symmetric ZDC z-profile gaussian, centered at z = 0, if model is
implimented correctly.\\
\textbf{Progress:} \\
I found the problem in the code - the difference between my method, and
Amaresh's method was how we handled normalization. As we know, gaussians have
normalization dependant on the width of the distriubtion, but this width gains
additional z-dependance when considering the $\beta^*$ squeeze effect. I
implimented this, and now distributions match very closely.\\
\textbf{New:}
Last time, I mentioned that multiple-collisions do not effect the resultant
distributions. I was wrong - so this parameter has been added back into the
simulation.
\end{frame}

\begin{frame}
	\frametitle{Parameter Space}
	\begin{itemize}
		\item Since I have not done the multiple collisions correction myself, I
			use the Run 15 numbers, and create a graph of scan-step vs multiple
			collisions rate, from which I interpolate the rate. To account for
			luminosity shifts, I allow the parameter to vary by a factor of 50\%.
		\item Care should be taken, as Run 15 had a higher average luminosity by a
			factor of 2, with respect to Run 12.
		\item Because distributions are generated randomly, there is some
			fluctuation in the final spectra. Therefore, we not halt the simualtion
			after 10 iterations, Which corresponds to the binary search step reaching
			a size of less than 1\% of the value of the original seed parameter.
	\end{itemize}
\end{frame}

